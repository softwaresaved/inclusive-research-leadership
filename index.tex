% Options for packages loaded elsewhere
\PassOptionsToPackage{unicode}{hyperref}
\PassOptionsToPackage{hyphens}{url}
\PassOptionsToPackage{dvipsnames,svgnames,x11names}{xcolor}
%
\documentclass[
  letterpaper,
  DIV=11,
  numbers=noendperiod]{scrreprt}
\usepackage{amsmath,amssymb}
\usepackage{lmodern}
\usepackage{iftex}
\ifPDFTeX
  \usepackage[T1]{fontenc}
  \usepackage[utf8]{inputenc}
  \usepackage{textcomp} % provide euro and other symbols
\else % if luatex or xetex
  \usepackage{unicode-math}
  \defaultfontfeatures{Scale=MatchLowercase}
  \defaultfontfeatures[\rmfamily]{Ligatures=TeX,Scale=1}
\fi
% Use upquote if available, for straight quotes in verbatim environments
\IfFileExists{upquote.sty}{\usepackage{upquote}}{}
\IfFileExists{microtype.sty}{% use microtype if available
  \usepackage[]{microtype}
  \UseMicrotypeSet[protrusion]{basicmath} % disable protrusion for tt fonts
}{}
\makeatletter
\@ifundefined{KOMAClassName}{% if non-KOMA class
  \IfFileExists{parskip.sty}{%
    \usepackage{parskip}
  }{% else
    \setlength{\parindent}{0pt}
    \setlength{\parskip}{6pt plus 2pt minus 1pt}}
}{% if KOMA class
  \KOMAoptions{parskip=half}}
\makeatother
\usepackage{xcolor}
\IfFileExists{xurl.sty}{\usepackage{xurl}}{} % add URL line breaks if available
\IfFileExists{bookmark.sty}{\usepackage{bookmark}}{\usepackage{hyperref}}
\hypersetup{
  pdftitle={Inclusive Research Leadership},
  pdfauthor={Neil Chue Hong, Tracy Teal},
  colorlinks=true,
  linkcolor={blue},
  filecolor={Maroon},
  citecolor={Blue},
  urlcolor={Blue},
  pdfcreator={LaTeX via pandoc}}
\urlstyle{same} % disable monospaced font for URLs
\usepackage{longtable,booktabs,array}
\usepackage{calc} % for calculating minipage widths
% Correct order of tables after \paragraph or \subparagraph
\usepackage{etoolbox}
\makeatletter
\patchcmd\longtable{\par}{\if@noskipsec\mbox{}\fi\par}{}{}
\makeatother
% Allow footnotes in longtable head/foot
\IfFileExists{footnotehyper.sty}{\usepackage{footnotehyper}}{\usepackage{footnote}}
\makesavenoteenv{longtable}
\usepackage{graphicx}
\makeatletter
\def\maxwidth{\ifdim\Gin@nat@width>\linewidth\linewidth\else\Gin@nat@width\fi}
\def\maxheight{\ifdim\Gin@nat@height>\textheight\textheight\else\Gin@nat@height\fi}
\makeatother
% Scale images if necessary, so that they will not overflow the page
% margins by default, and it is still possible to overwrite the defaults
% using explicit options in \includegraphics[width, height, ...]{}
\setkeys{Gin}{width=\maxwidth,height=\maxheight,keepaspectratio}
% Set default figure placement to htbp
\makeatletter
\def\fps@figure{htbp}
\makeatother
\setlength{\emergencystretch}{3em} % prevent overfull lines
\providecommand{\tightlist}{%
  \setlength{\itemsep}{0pt}\setlength{\parskip}{0pt}}
\setcounter{secnumdepth}{5}
% Make \paragraph and \subparagraph free-standing
\ifx\paragraph\undefined\else
  \let\oldparagraph\paragraph
  \renewcommand{\paragraph}[1]{\oldparagraph{#1}\mbox{}}
\fi
\ifx\subparagraph\undefined\else
  \let\oldsubparagraph\subparagraph
  \renewcommand{\subparagraph}[1]{\oldsubparagraph{#1}\mbox{}}
\fi
\newlength{\cslhangindent}
\setlength{\cslhangindent}{1.5em}
\newlength{\csllabelwidth}
\setlength{\csllabelwidth}{3em}
\newlength{\cslentryspacingunit} % times entry-spacing
\setlength{\cslentryspacingunit}{\parskip}
\newenvironment{CSLReferences}[2] % #1 hanging-ident, #2 entry spacing
 {% don't indent paragraphs
  \setlength{\parindent}{0pt}
  % turn on hanging indent if param 1 is 1
  \ifodd #1
  \let\oldpar\par
  \def\par{\hangindent=\cslhangindent\oldpar}
  \fi
  % set entry spacing
  \setlength{\parskip}{#2\cslentryspacingunit}
 }%
 {}
\usepackage{calc}
\newcommand{\CSLBlock}[1]{#1\hfill\break}
\newcommand{\CSLLeftMargin}[1]{\parbox[t]{\csllabelwidth}{#1}}
\newcommand{\CSLRightInline}[1]{\parbox[t]{\linewidth - \csllabelwidth}{#1}\break}
\newcommand{\CSLIndent}[1]{\hspace{\cslhangindent}#1}
\KOMAoption{captions}{tableheading}
\makeatletter
\makeatother
\makeatletter
\@ifpackageloaded{caption}{}{\usepackage{caption}}
\AtBeginDocument{%
\renewcommand*\contentsname{Table of contents}
\renewcommand*\listfigurename{List of Figures}
\renewcommand*\listtablename{List of Tables}
\renewcommand*\figurename{Figure}
\renewcommand*\tablename{Table}
}
\@ifpackageloaded{float}{}{\usepackage{float}}
\floatstyle{ruled}
\@ifundefined{c@chapter}{\newfloat{codelisting}{h}{lop}}{\newfloat{codelisting}{h}{lop}[chapter]}
\floatname{codelisting}{Listing}
\newcommand*\listoflistings{\listof{codelisting}{List of Listings}}
\makeatother
\makeatletter
\@ifpackageloaded{caption}{}{\usepackage{caption}}
\@ifpackageloaded{subcaption}{}{\usepackage{subcaption}}
\makeatother
\makeatletter
\@ifpackageloaded{tcolorbox}{}{\usepackage[many]{tcolorbox}}
\makeatother
\makeatletter
\@ifundefined{shadecolor}{\definecolor{shadecolor}{rgb}{.97, .97, .97}}
\makeatother
\makeatletter
\makeatother
\ifLuaTeX
  \usepackage{selnolig}  % disable illegal ligatures
\fi

\title{Inclusive Research Leadership}
\author{Neil Chue Hong, Tracy Teal}
\date{4/2/2022}

\begin{document}
\maketitle

\ifdefined\Shaded\renewenvironment{Shaded}{\begin{tcolorbox}[boxrule=0pt, enhanced, frame hidden, sharp corners, interior hidden, borderline west={3pt}{0pt}{shadecolor}]}{\end{tcolorbox}}\fi

\renewcommand*\contentsname{Table of contents}
{
\hypersetup{linkcolor=}
\setcounter{tocdepth}{2}
\tableofcontents
}
\hypertarget{preface}{%
\chapter*{Preface}\label{preface}}
\addcontentsline{toc}{chapter}{Preface}

These are materials that could be used in an inclusive research
leadership workshop. These are currently under development in this
repository
\url{https://github.com/tracykteal/inclusive-research-leadership}.

This content is written in Quarto. To learn more about Quarto books
visit \url{https://quarto.org/docs/books}.

\hypertarget{about-this-content}{%
\subsection*{About this content}\label{about-this-content}}
\addcontentsline{toc}{subsection}{About this content}

Researchers and research software developers have developed experience
and expertise in their areas of work. They have spent time learning how
to code, analyze data and are experts in their domains. As people become
leaders in their field, whether it's a small group or a large team, they
also need to develop leadership and management skills, but often haven't
had the opportunity to learn them, or are self-taught. This lack of
knowledge around team leadership negatively impacts the person in the
leadership position as well as those in their team. Therefore there is
the opportunity for short-format, practical, hands-on training for
people in or transitioning to research leadership roles.

We know a lot now from research around leadership as to what makes
effective leadership, meaning leadership that allows a team to do its
best work, both for the individuals on the team, and the team itself.
That includes elements of creating psychological safety and providing
opportunities for mastery, autonomy and purpose, and centering
inclusiveness, accessibility and culturally responsive practices.
Overall, what we know makes for effective leadership, is not always how
we see leadership practiced. Therefore not only are there not learning
opportunities, but what people learn by `watching' are not effective
practices.

We are developing a course on Inclusive Research Leadership modeled on
the The Carpentries 2-day workshop format, which aims to provide
participants with opportunities to learn about leadership, based on what
we know works, and that values people - both the leader themselves and
the people they lead.

\hypertarget{introduction}{%
\chapter*{Introduction}\label{introduction}}
\addcontentsline{toc}{chapter}{Introduction}

This the beginning of a set of content for an inclusive leadership
workshop.

Ideas for this curriculum are currently being discussed at an SSI CW22
workshop and we're planning a sprint to develop scenarios that could be
used as exercises in a workshop.

\hypertarget{feedback-welcome}{%
\subsection*{Feedback welcome!}\label{feedback-welcome}}
\addcontentsline{toc}{subsection}{Feedback welcome!}

This is content that is still coming together and ideas and feedback are
welcome! Add issues in
\href{https://github.com/tracykteal/inclusive-research-leadership}{this
repo} or get in touch with Tracy Teal or Neil Chue Hong.

\part{Scenarios}

For a leadership workshop, the excercises are scenarios, where people
can practice going through them, either reading and responding
themselves, or practicing with a partner.

We'd like to develop scenarios that give workshop participants the
chance to practice the things they have just learned in a module. For
example, for the `Giving and receiving feedback' module, you would want
practice scenarios where you could practice giving feedback to someone
and practice receiving difficult feedback.

Like any skill, only practice helps you get better! These scenarios in a
workshop setting give you the space for that practice.

\hypertarget{scenario-format}{%
\section*{Scenario format}\label{scenario-format}}
\addcontentsline{toc}{section}{Scenario format}

Each scenario has its own page. This is an example.

For each scenario include:

\textbf{Skill}: List the skill being practiced\\
\textbf{Format}: Is this an individual, paired or group exercise\\
\textbf{Directions}: Are there any particular directions?\\
\textbf{Time:}: How much time should participants spend on this
exercise? How much on the scenario and how much on the reflection
questions?\\
\textbf{Scenario}: The story of the scenario\\
\textbf{Reflection questions}: What questions might you ask yourself or
others after the scenario?

\hypertarget{creating-a-new-scenario}{%
\subsection*{Creating a new scenario}\label{creating-a-new-scenario}}
\addcontentsline{toc}{subsection}{Creating a new scenario}

To create a new scenario

\begin{itemize}
\tightlist
\item
  Clone repo
\item
  Copy the file scenario\_example.qmd to scenario\_NEWNAME.qmd
\item
  Edit scenario\_NEWNAME.qmd (editing instructions)
\item
  Add the name of the new file to \_quarto.yml
\item
  Push or put in a pull request to repo
\item
  You will see the file now included at URL
\end{itemize}

\hypertarget{editing-the-scenario.qmd-files}{%
\subsection*{Editing the scenario.qmd
files}\label{editing-the-scenario.qmd-files}}
\addcontentsline{toc}{subsection}{Editing the scenario.qmd files}

The files for this book are written in
\href{https://quarto.org/}{Quarto}. .qmd files are like R Markdown
files. You can edit them in any text editor, or in the RStudio IDE.

Instructions on how to work with Quarto books locally is here:
\url{https://quarto.org/docs/books/}

\hypertarget{scenario-template}{%
\chapter*{Scenario Template}\label{scenario-template}}
\addcontentsline{toc}{chapter}{Scenario Template}

This is template for creating a scenario. To create a new scenario, copy
this file to a new file in this same directory `scenarios/' then fill in
your own content in the different sections.

\hypertarget{title-of-scenario}{%
\section*{Title of Scenario}\label{title-of-scenario}}
\addcontentsline{toc}{section}{Title of Scenario}

What is the title of this scenario? Use the module name and then some
descriptor for the exercise.

\hypertarget{skill}{%
\subsection*{Skill}\label{skill}}
\addcontentsline{toc}{subsection}{Skill}

What skill are you practicing?

\hypertarget{format}{%
\subsection*{Format}\label{format}}
\addcontentsline{toc}{subsection}{Format}

Is this an individual, paired or group exercise?

\hypertarget{directions}{%
\subsection*{Directions}\label{directions}}
\addcontentsline{toc}{subsection}{Directions}

Are there any particular directions for this exercise?

\hypertarget{time}{%
\subsection*{Time}\label{time}}
\addcontentsline{toc}{subsection}{Time}

How much time should participants spend on this exercise? How much on
the scenario and how much on the reflection questions?

\hypertarget{things-to-remember}{%
\subsection*{Things to remember}\label{things-to-remember}}
\addcontentsline{toc}{subsection}{Things to remember}

Are there quick tip things to remember from the lesson in this scenario?

\hypertarget{scenario}{%
\subsection*{Scenario}\label{scenario}}
\addcontentsline{toc}{subsection}{Scenario}

Write the story of the scenario. It may just be for one person, or you
may write `parts' for 2 or more people.

Person 1:

Person 2:

\hypertarget{reflection-questions}{%
\subsection*{Reflection questions}\label{reflection-questions}}
\addcontentsline{toc}{subsection}{Reflection questions}

What questions might you ask yourself or others after the scenario?

\hypertarget{example-scenario}{%
\chapter*{Example Scenario}\label{example-scenario}}
\addcontentsline{toc}{chapter}{Example Scenario}

This is an example scenario. To create a new scenario, copy this file to
a new file in this same directory `scenarios/' then fill in your own
content in the different sections.

\hypertarget{feedback-scenario-1}{%
\section*{Feedback: Scenario 1}\label{feedback-scenario-1}}
\addcontentsline{toc}{section}{Feedback: Scenario 1}

\hypertarget{skill-1}{%
\subsection*{Skill}\label{skill-1}}
\addcontentsline{toc}{subsection}{Skill}

The skill being practiced is giving feedback.

\hypertarget{format-1}{%
\subsection*{Format}\label{format-1}}
\addcontentsline{toc}{subsection}{Format}

2 person

\hypertarget{directions-1}{%
\subsection*{Directions}\label{directions-1}}
\addcontentsline{toc}{subsection}{Directions}

In this scenario one person will be the one giving the feedback and the
other will be the one receiving the feedback. Both people should read
the scenario.

\hypertarget{time-1}{%
\subsection*{Time}\label{time-1}}
\addcontentsline{toc}{subsection}{Time}

Spend 3 minutes giving the feedback and responding, 2 minutes
discussing.

\hypertarget{things-to-remember-1}{%
\subsection*{Things to remember}\label{things-to-remember-1}}
\addcontentsline{toc}{subsection}{Things to remember}

Remember: You want to provide feedback, so that the student is aware of
the issues and can work with you on finding solutions. You don't want to
provide solutions for the student.

\hypertarget{scenario-1}{%
\subsection*{Scenario}\label{scenario-1}}
\addcontentsline{toc}{subsection}{Scenario}

Person 1: You are a professor leading a lab of 3 students and 2
postdocs. A third year student in the lab has been been working on a
project for awhile and not making progress. The student is frustrated
with their lack of progress, and you have to report out on the findings
from this grant in 6 months in order to continue receiving funding. In
your regular meetings with your student you notice that they keep using
the same approaches and seem reluctant to try new approaches.

Person 2: You are a third year graduate student. You've been working on
a project for awhile and are frustrated you're not making progress. You
are embarrassed though that you haven't made more progress and don't
want to look bad to your professor or lab mates, but don't know what to
do to change things.

\hypertarget{reflection-questions-1}{%
\subsection*{Reflection questions}\label{reflection-questions-1}}
\addcontentsline{toc}{subsection}{Reflection questions}

Person 2: How did that feedback feel to you?\\
Person 1: How did giving the feedback feel to you? Anything you would
have done differently?

\hypertarget{references}{%
\chapter*{References}\label{references}}
\addcontentsline{toc}{chapter}{References}

References will be added.

\hypertarget{refs}{}
\begin{CSLReferences}{0}{0}
\end{CSLReferences}

\end{document}
